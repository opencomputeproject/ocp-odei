%!TEX root =Metcalfe+Boggs.tex
% This is what ChatGPT came up with for Table 1 3 in the Metcalfe + Boggs Paper

\begin{table}[ht]
\centering \tiny % \small
\caption*{Table 1. Ethernet Efficiency}
%\label{tab:ethernet-efficiency}
\begin{tabular}{c|cccc}
\hline
\textbf{Q} & \textbf{P = 4096} & \textbf{P = 1024} & \textbf{P = 512} & \textbf{P = 48} \\
\hline
1   & 1.0000 & 1.0000 & 1.0000 & 1.0000 \\
2   & 0.9884 & 0.9552 & 0.9143 & 0.5000 \\
3   & 0.9857 & 0.9447 & 0.8951 & 0.4444 \\
4   & 0.9842 & 0.9396 & 0.8862 & 0.4219 \\
5   & 0.9834 & 0.9367 & 0.8810 & 0.4096 \\
10  & 0.9818 & 0.9310 & 0.8709 & 0.3874 \\
32  & 0.9807 & 0.9272 & 0.8642 & 0.3737 \\
64  & 0.9805 & 0.9263 & 0.8627 & 0.3708 \\
128 & 0.9804 & 0.9259 & 0.8620 & 0.3693 \\
256 & 0.9803 & 0.9257 & 0.8616 & 0.3686 \\
\hline
\end{tabular}
\end{table}

%%[[Graph from Above Table  -- THIS IS NOT IN ORIGINAL PAPER]]
%
%\begin{figure}[ht]
%    \centering \small
%    \begin{tikzpicture}
%    \begin{axis}[
%        width=0.5\textwidth,
%        xlabel={\(\displaystyle Q\)},
%        ylabel={Efficiency},
%        xmin=0, xmax=260,
%        ymin=0.3, ymax=1.05,
%        legend pos=south west,
%        grid=both
%    ]
%    
%    %-- P = 4096
%    \addplot[
%        mark=o,
%        blue
%    ]
%    coordinates {
%        (1,1.0000)
%        (2,0.9884)
%        (3,0.9857)
%        (4,0.9842)
%        (5,0.9834)
%        (10,0.9818)
%        (32,0.9807)
%        (64,0.9805)
%        (128,0.9804)
%        (256,0.9803)
%    };
%    \addlegendentry{P = 4096}
%    
%    %-- P = 1024
%    \addplot[
%        mark=square,
%        red
%    ]
%    coordinates {
%        (1,1.0000)
%        (2,0.9552)
%        (3,0.9447)
%        (4,0.9396)
%        (5,0.9367)
%        (10,0.9310)
%        (32,0.9272)
%        (64,0.9263)
%        (128,0.9259)
%        (256,0.9257)
%    };
%    \addlegendentry{P = 1024}
%    
%    %-- P = 512
%    \addplot[
%        mark=triangle,
%        brown
%    ]
%    coordinates {
%        (1,1.0000)
%        (2,0.9143)
%        (3,0.8951)
%        (4,0.8862)
%        (5,0.8810)
%        (10,0.8709)
%        (32,0.8642)
%        (64,0.8627)
%        (128,0.8620)
%        (256,0.8616)
%    };
%    \addlegendentry{P = 512}
%    
%    %-- P = 48
%    \addplot[
%        mark=*,
%        green!60!black
%    ]
%    coordinates {
%        (1,1.0000)
%        (2,0.5000)
%        (3,0.4444)
%        (4,0.4219)
%        (5,0.4096)
%        (10,0.3874)
%        (32,0.3737)
%        (64,0.3708)
%        (128,0.3693)
%        (256,0.3686)
%    };
%    \addlegendentry{P = 48}
%    
%    \end{axis}
%    \end{tikzpicture}
%%    \caption{Ethernet Efficiency vs.\ Q for Different Packet Sizes \(P\)}
%    \label{fig:ethernet-efficiency-plot}
%\end{figure}
